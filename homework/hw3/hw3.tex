%-------------------------------------------------------------------------------
%   PACKAGES AND OTHER DOCUMENT CONFIGURATIONS
%-------------------------------------------------------------------------------
% Compile using pdflatex -shell-escape hw3.tex
%
\documentclass[paper=a4, fontsize=11pt]{scrartcl} % A4 paper and 11pt font size
\usepackage{fancyhdr} % Required for custom headers
\usepackage{lastpage} % Required to determine the last page for the footer
\usepackage{extramarks} % Required for headers and footers
\usepackage[usenames,dvipsnames]{color} % Required for custom colors
\usepackage{graphicx} % Required to insert images
\usepackage{listings} % Required for insertion of code
\usepackage{courier} % Required for the courier font
\usepackage[T1]{fontenc} % Use 8-bit encoding that has 256 glyphs
\usepackage[english]{babel} % English language/hyphenation
\usepackage{amsmath,amsfonts,amsthm} % Math packages
\usepackage{minted} % code highlighting

\usepackage{sectsty} % Allows customizing section commands
\allsectionsfont{\centering \normalfont\scshape} % Make all sections centered, 
                                                 % the default font and small 
                                                 % caps

\pagestyle{fancyplain} % Makes all pages in the document conform to the custom
                       % headers and footers
\fancyhead{} % No page header - if you want one, create it in the same way as 
             % the footers below
\fancyfoot[L]{} % Empty left footer
\fancyfoot[C]{} % Empty center footer
\fancyfoot[R]{\thepage} % Page numbering for right footer
\renewcommand{\headrulewidth}{0pt} % Remove header underlines
\renewcommand{\footrulewidth}{0pt} % Remove footer underlines
\setlength{\headheight}{13.6pt} % Customize the height of the header

\numberwithin{equation}{section} % Number equations within sections (i.e. 1.1, 
                                 % 1.2, 2.1, 2.2 instead of 1, 2, 3, 4)
\numberwithin{figure}{section} % Number figures within sections (i.e. 1.1, 1.2,
                               % 2.1, 2.2 instead of 1, 2, 3, 4)
\numberwithin{table}{section} % Number tables within sections (i.e. 1.1, 1.2, 
                              % 2.1, 2.2 instead of 1, 2, 3, 4)

\setlength\parindent{0pt} % Removes all indentation from paragraphs - comment 
                          % this line for an assignment with lots of text

%-------------------------------------------------------------------------------
%   TITLE SECTION
%-------------------------------------------------------------------------------

\newcommand{\horrule}[1]{\rule{\linewidth}{#1}} % horizontal cmd, arg = height
\newcommand{\name}{Colin Bradford} % student name
\newcommand{\hwnum}{3} % homework number
\newcommand{\classnum}{CS 325} % class num with abreviation
\newcommand{\classname}{Analysis of Algorithms} % name of class
\newcommand{\hwtitle}{\classnum: Homework \hwnum}
\newtheorem{theorem}{Theorem}[section]

\title{ 
    \normalfont \normalsize 
    \textsc{Oregon State University} \\ [25pt]
    \horrule{0.5pt} \\[0.4cm] % Thin top horizontal rule
    \huge \hwtitle \\ % The assignment title
    \horrule{2pt} \\[0.5cm] % Thick bottom horizontal rule
}

\author{\name} % Your name

\date{\normalsize\today} % Today's date or a custom date

%-------------------------------------------------------------------------------
%   DOCUMENT
%-------------------------------------------------------------------------------
\begin{document}

\maketitle % Print the title

\begin{enumerate}

% ------------------------------------------------------------------------------    
%   PROBLEM 1
% ------------------------------------------------------------------------------
    \item \hfill \\
    \begin{enumerate}
        \item $T(n) = \Theta(n^2)$
        \begin{proof}
            Given $T(n) = T(n - 2) + n$ \\
            Let $k$ be the number of iterations for any given $n$ \\
            Since the size $n$ is reduce by 2 each iteration, then after some 
            $k$ iterations $n = 1$ which will be the $T(1)$ case. Thus we know
            that $n - 2k = 1$. Rewritten the equation looks like 
            \begin{equation}
                k = \frac{ n - 1 }{ 2 }
            \end{equation}
            Next it will be shown that $T(n) \approx n^2$ \\
            Given the arithmetic series
            \begin{equation}
                \sum\limits_{i=1}^k i = \frac{(k)(k + 1)}{2}
            \end{equation}
            then using the iterative method we can show
            \begin{align*}
                T(n) &= T(n - 2) + n \\
                T(n - 2) &= T(n - 4) + n - 2 \\
                \ldots &= n +(n - 2 +(n - 4 + (\ldots))) \\
                &= (n + n + \ldots + n) + (- 2 - 4 - \ldots - 2k) \\
                &= kn + (-2)\sum\limits_{i=1}^k i \\
                &= kn - 2\left(\frac{(k)(k + 1)}{2}\right) \tag{eq. 0.2} \\
                &= kn - (k)(k + 1) \\
                &= \left(\frac{ n - 1 }{ 2 }\right)\left(n\right) - 
                \left(\frac{ n - 1 }{ 2 }\right)\left(\frac{ n - 1 }{ 2 } + 
                \frac{ 2 }{ 2 }\right)  \tag{eq. 0.1}\\
                &= \frac{ n^2 - n }{ 2 } - 
                \left(\frac{ n - 1 }{ 2 }\right)\left(
                \frac{ n + 1 }{ 2 }\right) \\
                &= \frac{ n^2 - n }{ 2 } - \frac{ n^2 - 1 }{ 4 } \\
                &= \frac{ 2n^2 - 2n }{ 4 } - \frac{ n^2 - 1 }{ 4 } \\
                T(n) &= \frac{ n^2 - 2n + 1 }{ 4 } \\
            \end{align*}
            Thus we can see that $T(n)$ must be $\Theta(n^2)$.
        \end{proof}

    \item $T(n) = \Theta(n)$
    \begin{proof}
        Given $T(n) = T(n - 1) + 3$ \\
        Let $k$ be the number of iterations for any given $n$ \\
        Since the size $n$ is reduce by 1 each iteration, then after some 
        $k$ iterations $n = 1$ which will be the $T(1)$ case. Thus we know
        that $n - k = 0$ assuming we don't care about the $T(1)$ case. 
        Rewritten the equation looks like 
        \begin{equation}
            k = n
        \end{equation}
        Using the iterative method we can show
        \begin{align*}
            T(n) = T(n - 1) + 3 \\
            T(n - 1) = T(n - 2) + 3 \\
            \ldots = 3 + (3 + (3 + (\ldots))) \\
            T(n) = 3k \\
            T(n) = 3n \tag{eq 0.3} \\
        \end{align*}
        Thus we know that $T(n)$ must be $\Theta(n)$.
    \end{proof}

    \item $T(n) = \Theta(n)$
    \begin{proof}
        Given $T(n) = 2T(\frac{ n }{ 4 }) + n$ \\
        \begin{theorem}[Master's theorem]
            Let $a \in \mathbb{Z}$ and $a \geq 1$, and $b \in \mathbb{R}$ and 
            $b \geq 1$. Let $c \in \mathbb{R}$ and $c > 0$, and 
            $d \in \mathbb{R}$ and $d \geq 0$. Given a recurrence of the form
        \[
            T(n) = 
            \begin{cases}
                aT(n/b) n^c, & \text{if $n > 1$}.\\
                d, & \text{if $n = 1$}.
            \end{cases} 
        \]
        then for n a power of b,
        \begin{enumerate}
            \item if $\log_b{a} < c$, $T(n) = \Theta\left(n^c\right)$
            \item if $\log_b{a} = c$, $T(n) = \Theta\left(n^c\log{n}\right)$
            \item if $\log_b{a} > c$, $T(n) = \Theta\left(n^{ \log_b{a} }\right)$
        \end{enumerate}
        \end{theorem}
        Using the Master's theorem, let $a = 2$, $b = 4$, and $c = 1$ \\
        Assume, $\log_b{a} < c$ and as a result $T(n) = \Theta\left(n^c\right)$
        \begin{align*}
            \log_4{2} &< 1 \\
            4^{\log_4{2}} &< 4^1 \\
            2 &< 4
        \end{align*}
        Thus we know that $T(n) = \Theta\left(n^1\right) = \Theta(n)$.
    \end{proof}

    \item $T(n) = \Theta(n^{\frac{ 5 }{ 2 }})$
    \begin{proof}
        Given $T(n) = 4T(\frac{ n }{ 2 }) + n^2\sqrt{n}$ \\
        We can rewrite this equation to be
        \[ 4T(\frac{ n }{ 2 }) + n^{5/2} \]
        Using the Master's theorem, $a = 4$, $b = 2$, and $c = \frac{5}{2}$ \\
        Assume, $\log_b{a} < c$ and as a result $T(n) = \Theta\left(n^c\right)$
        \begin{align*}
            \log_2{4} & < \frac{5}{2} \\
            2 & < \frac{5}{2} \\
            \frac{4}{2} & < \frac{5}{2} \\
            4 & < 5
        \end{align*}
        Thus we know that $T(n) = \Theta\left(n^{5/2}\right)$.
    \end{proof}
    \end{enumerate}

% ------------------------------------------------------------------------------
%   PROBLEM 2
% ------------------------------------------------------------------------------
    \item I would choose the $f_C(n)$ algorithm. \\
    \begin{enumerate}
        \item $T(n) = \Theta\left(n^{\log_2{5}}\right)$ \\
        Given the description $T(n) = 5T\left(\frac{ n }{ 2 }\right) + n$ \\
        Using the Master's theorem, $a = 5$, $b = 2$, and $c = 1$ \\
        Assume, $\log_b{a} > c$ and as a result 
        $T(n) = \Theta\left(n^{\log_b{a}}\right)$
        \begin{align*}
            \log_2{5} & > 1 \\
            2^{\log_2{5}} & > 2^1 \\
            5 > 2
        \end{align*}
        Thus $T(n) = \Theta\left(n^{\log_2{5}}\right)$

        \item $T(n) = \Theta\left(2^n\right)$ \\
        Given the description $T(n) = 2T\left(n - 1\right) + c$ \\
        Using the iterative method
        \begin{align*}
            T(n) & = 2T\left(n - 1\right) + c \\
            T(n - 1) & = 2T\left(n - 2\right) + c \\
            \ldots & = c + 2(c + 2(c + 2(\ldots))) \\
            & = c\sum\limits_{i=0}^k 2^i \\
            & = c\left(\frac{ 1 - 2^k}{ 1 - 2 }\right) \\
            & = c\left(\frac{ 1 - 2^k}{ -1 }\right) \\
            & = -c\left(1 - 2^k\right) \\
            & = -c + (c)(2^k) \\
            & = -c + (c)(2^n) \\
        \end{align*}
        Thus $T(n) = \Theta\left(2^n\right)$

        \item $T(n) = \Theta\left(n^2\log{n}\right)$ \\
        Given the description $T(n) = 9T\left(\frac{ n }{ 3 }\right) + n^2$ \\
        Using the Master's theorem, $a = 9$, $b = 3$, and $c = 2$ \\
        Assume, $\log_b{a} = c$ and as a result 
        $T(n) = \Theta\left(n^c\log{n}\right)$
        \begin{align*}
            \log_3{9} & = 2 \\
            3^{\log_3{9}} & = 3^2 \\
            9 & = 9
        \end{align*}
        Thus $T(n) = \Theta\left(n^2\log{n}}\right)$
    \end{enumerate}
    Comparing each algorithm's running time by limits will show that the fastest
    algorithm is algorithm C. \\
    Comparing B and C: 
    \[
        \lim_{n\to\infty} \frac{ f_B(n) }{ f_C(n) } = 
        \lim_{n\to\infty} \frac{ 2^n }{ n^2\log{n} }
    \]
    to find what this limit approaches we must take the derivative multiple
    times \\
    \begin{align*}
        \lim_{n\to\infty} \frac{ 2^n }{ n^2\log{n} } \\
        \lim_{n\to\infty} \left(2^n\ln{2}\right)\left(
            \frac{ \ln{10} }{ n }\right) \\
        \lim_{n\to\infty} \frac{ 2^n\ln{10}\ln{2} }{ n } \tag{$f'(n)$} \\
        \lim_{n\to\infty} \frac{ 2^n\ln{10}\ln{2} }{ 2n\ln{n} } 
            \tag{$f''(n)$} \\
        \lim_{n\to\infty} \frac{ 2^n }{ \log{n} } \tag{$f'''(n)$} \\
        \lim_{n\to\infty} 2^nn = \infty \tag{$f''''(n)$}
    \end{align*}
    Thus since $f_B(n)$ approaches $\infty$ faster that $f_C(n)$ we can say
    that $f_C(n)$ is faster than $f_B(n)$. \\
    Next we compare C and A: 
    \[
        \lim_{n\to\infty} \frac{ f_A(n) }{ f_C(n) } = 
        \lim_{n\to\infty} \frac{ n^{ \log_2{5} } }{ n^2\log{n} }
    \]
    and find after take the derivative that:
    \[
        \lim_{n\to\infty} n^{ \log_2{5} - 2}\ln{10} = \infty
    \]
    ($\log_2{5} > 2$ thus $\log_2{5} - 2 > 0$) \\
    Thus we can conclude that $f_C(n)$ is the most efficient algorithm.

% ------------------------------------------------------------------------------
%   PROBLEM 3
% ------------------------------------------------------------------------------
    \item $T(n) = \Theta(n)$ \\
    The equation for number of lines printed is given by
    \[ T(n) = 2T\left(\frac{ n }{ 2 }\right) + 1 \]
    Assume $n = 2^k$, which can be written as $k = \lg{n}$ \\
    Thus using the iterative method:
    \begin{align*}
        T(n) & = 2T\left(\frac{ n }{ 2 }\right) + 1 \\
        T\left(\frac{ n }{ 2 }\right) & = 2T\left(\frac{ n }{ 4 }\right) + 1 \\
        \ldots & = 1 + 2(1 + 2(1 + 2(\ldots))) \\
        & = \sum\limits_{i=0}^k 2^k \\
        & = \frac{ 1 - 2^k }{ 1 - 2 } \\
        & = 2^k - 1 \\
        & = 2^{\lg{n}} - 1 \\
        & = n - 1
    \end{align*}
    Thus $T(n) = \Theta(n)$.
% ------------------------------------------------------------------------------
%   PROBLEM 4
% ------------------------------------------------------------------------------
    \item \hfill \\
    \begin{minted}[mathescape,
                   linenos,
                   numbersep=5pt,
                   gobble=2,
                   frame=lines,
                   framesep=2mm]{python}
    Ternary-Sort(sTerm, list, low, high):
        n = high - low
        lowMid = n / 3 + low
        highMid = (2 * n) / 3 + low

        if (sTerm == list[lowMid]):
            return lowMid
        else if(sTerm == list[highMid]):
            return highMid
        else if (n == 0):
            return -1
        else if (sTerm < list[lowMid]):
            return Ternary-Sort(sTerm, list, low, lowMid)
        else if (sTerm < list[highMid]):
            return Ternary-Sort(sTerm, list, lowMid, highMid)
        else
            return Ternary-Sort(sTerm, list, highMid, high)
    \end{minted}
    \begin{description}
        \item[binary] $T(n) = T\left(\frac{ n }{ 2 }\right) + c$ \\
        Assume $n = 2^k$ \\
        Using the iterative method:
        \begin{align*}
            T(n) & = T\left(\frac{ n }{ 2 }\right) + c \\
            T\left(\frac{ n }{ 2 }\right) & = T\left(\frac{ n }{ 4 }\right) + c \\
            \ldots & = c + (c + (c + (\ldots))) \\
            & = ck \\
            T(n) = c\lg{n}
        \end{align*}
        Thus $T(n) = \Theta(\lg{n})$

        \item[ternary]$T(n) = T\left(\frac{ n }{ 3 }\right) + c$ 
        Assume $n = 3^k$ \\
        Using the iterative method:
        \begin{align*}
            T(n) & = T\left(\frac{ n }{ 3 }\right) + c \\
            T\left(\frac{ n }{ 3 }\right) & = T\left(\frac{ n }{ 9 }\right) + c \\
            \ldots & = c + (c + (c + (\ldots))) \\
            & = ck \\
            T(n) = c\log_3{n}
        \end{align*}
        Thus $T(n) = \Theta(\log_3{n})$ which is equivalent to $\Theta(\lg{n})$.
    \end{description}
    Both binary and ternary search have the same relative running time.
% ------------------------------------------------------------------------------
%   PROBLEM 5
% ------------------------------------------------------------------------------
    \item \hfill \\
\end{enumerate}
\end{document}
%-------------------------------------------------------------------------------


