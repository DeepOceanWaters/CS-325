%-------------------------------------------------------------------------------
%   PACKAGES AND OTHER DOCUMENT CONFIGURATIONS
%-------------------------------------------------------------------------------
\documentclass[paper=a4, fontsize=11pt]{scrartcl} % A4 paper and 11pt font size
\usepackage{fancyhdr} % Required for custom headers
\usepackage{lastpage} % Required to determine the last page for the footer
\usepackage{extramarks} % Required for headers and footers
\usepackage[usenames,dvipsnames]{color} % Required for custom colors
\usepackage{graphicx} % Required to insert images
\usepackage{listings} % Required for insertion of code
\usepackage{courier} % Required for the courier font
\usepackage[T1]{fontenc} % Use 8-bit encoding that has 256 glyphs
\usepackage[english]{babel} % English language/hyphenation
\usepackage{amsmath,amsfonts,amsthm} % Math packages
\usepackage{enumitem}
\usepackage{algorithm}
\usepackage{algpseudocode}

\usepackage{sectsty} % Allows customizing section commands
\allsectionsfont{\centering \normalfont\scshape} % Make all sections centered, 
                                                 % the default font and small 
                                                 % caps

\pagestyle{fancyplain} % Makes all pages in the document conform to the custom
                       % headers and footers
\fancyhead{} % No page header - if you want one, create it in the same way as 
             % the footers below
\fancyfoot[L]{} % Empty left footer
\fancyfoot[C]{} % Empty center footer
\fancyfoot[R]{\thepage} % Page numbering for right footer
\renewcommand{\headrulewidth}{0pt} % Remove header underlines
\renewcommand{\footrulewidth}{0pt} % Remove footer underlines
\setlength{\headheight}{13.6pt} % Customize the height of the header

\numberwithin{equation}{section} % Number equations within sections (i.e. 1.1, 
                                 % 1.2, 2.1, 2.2 instead of 1, 2, 3, 4)
\numberwithin{figure}{section} % Number figures within sections (i.e. 1.1, 1.2,
                               % 2.1, 2.2 instead of 1, 2, 3, 4)
\numberwithin{table}{section} % Number tables within sections (i.e. 1.1, 1.2, 
                              % 2.1, 2.2 instead of 1, 2, 3, 4)

\setlength\parindent{0pt} % Removes all indentation from paragraphs - comment 
                          % this line for an assignment with lots of text

%-------------------------------------------------------------------------------
%   TITLE SECTION
%-------------------------------------------------------------------------------

\newcommand{\horrule}[1]{\rule{\linewidth}{#1}} % horizontal cmd, arg = height
\newcommand{\name}{Charles Jenkins} % student name
\newcommand{\hwnum}{4} % homework number
\newcommand{\classnum}{CS 325} % class num with abreviation
\newcommand{\classname}{Analysis of Algorithms} % name of class
\newcommand{\hwtitle}{\classnum: Project \hwnum}

\title{ 
    \normalfont \normalsize 
    \textsc{Oregon State University} \\ [25pt]
    \large Project Group 21
    \horrule{0.5pt} \\[0.4cm] % Thin top horizontal rule
    \huge \hwtitle \\ % The assignment title
    \horrule{2pt} \\[0.5cm] % Thick bottom horizontal rule
}

\author{
	Charles Jenkins
	\and
    Albert Le
    \and
    Colin Bradford    
} % Your name

\date{\normalsize\today} % Today's date or a custom date

%-------------------------------------------------------------------------------
%   DOCUMENT
%-------------------------------------------------------------------------------
\begin{document}

\maketitle % Print the title

\section{The Ideas Behind the Algorithm}
Our program makes use of two different TSP algorithms in tandem to achieve better optimization for the tests. We start with a greedy algorithm and then feed its results into a 2-opt algorithm. We then take the results from the 2-opt algorithm and feed it back into the 2-opt algorithm over and over again until we either no longer find an improved route or we reach some time limit.\newline

The whole process works as follows:

\begin{itemize}

  \item Read the input file and build a set of coordinates for each city.
  \item Calculate the distances for every possible edge in the graph.  
  \item Build a 2D list of the edges as (distance, source, destination) triplets. Since this would be a symmetrical square nxn matrix, we only create the lower half of it to save processing time. We also do not include edges with a cost of zero (edges from a node to itself).
  \item Sort this list in ascending order of distances.
  \item Pass this list to the greedy algorithm.
  \item Pick the lowest cost edge.
  \item Add its source to our route, mark it as visited, and increase by one the degree of its source and destination vertices.
  \item While we haven't visited all the nodes, add to our route the next lowest cost edge whose source matches our previous edge's destination, ensuring that no vertex ever has a degree of more than two and that we form no cycles. Return the distance and route.
  \item Feed that resulting route into the 2-opt algorithm.
  \item Perform standard 2-opt swaps (reorder routes such that they do not cross over themselves), keeping track of any better solutions we find until either we find no more improvements or we reach some time limit. Return the distance and route.
  \item Continue feeding the resulting route of the 2-opt algorithm back into the 2-opt algorithm until either we find no more improvements or reach some time limit.
  \item Output to file our final distance and route solution.
  \item NOTE: On very large input sizes, like in \verb|tsp_example_3.txt|, we resort to only a 2-opt solution on a random starting path to ensure reasonable termination times (execution can be stopped midway and still offer a solution, unlike the greedy algorithm which requires a fully built and sorted table up front).

\end{itemize}

Acknowledgments for the basic ideas behind each algorithm can be found in the source code.\newline

\end{document}
%-------------------------------------------------------------------------------


