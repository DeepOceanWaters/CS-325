%-------------------------------------------------------------------------------
%   PACKAGES AND OTHER DOCUMENT CONFIGURATIONS
%-------------------------------------------------------------------------------
\documentclass[paper=a4, fontsize=11pt]{scrartcl} % A4 paper and 11pt font size
\usepackage{fancyhdr} % Required for custom headers
\usepackage{lastpage} % Required to determine the last page for the footer
\usepackage{extramarks} % Required for headers and footers
\usepackage[usenames,dvipsnames]{color} % Required for custom colors
\usepackage{graphicx} % Required to insert images
\usepackage{listings} % Required for insertion of code
\usepackage{courier} % Required for the courier font
\usepackage[T1]{fontenc} % Use 8-bit encoding that has 256 glyphs
\usepackage[english]{babel} % English language/hyphenation
\usepackage{amsmath,amsfonts,amsthm} % Math packages
\usepackage{enumitem}
\usepackage{algorithm}
\usepackage{algpseudocode}

\usepackage{sectsty} % Allows customizing section commands
\allsectionsfont{\centering \normalfont\scshape} % Make all sections centered, 
                                                 % the default font and small 
                                                 % caps

\pagestyle{fancyplain} % Makes all pages in the document conform to the custom
                       % headers and footers
\fancyhead{} % No page header - if you want one, create it in the same way as 
             % the footers below
\fancyfoot[L]{} % Empty left footer
\fancyfoot[C]{} % Empty center footer
\fancyfoot[R]{\thepage} % Page numbering for right footer
\renewcommand{\headrulewidth}{0pt} % Remove header underlines
\renewcommand{\footrulewidth}{0pt} % Remove footer underlines
\setlength{\headheight}{13.6pt} % Customize the height of the header

\numberwithin{equation}{section} % Number equations within sections (i.e. 1.1, 
                                 % 1.2, 2.1, 2.2 instead of 1, 2, 3, 4)
\numberwithin{figure}{section} % Number figures within sections (i.e. 1.1, 1.2,
                               % 2.1, 2.2 instead of 1, 2, 3, 4)
\numberwithin{table}{section} % Number tables within sections (i.e. 1.1, 1.2, 
                              % 2.1, 2.2 instead of 1, 2, 3, 4)

\setlength\parindent{0pt} % Removes all indentation from paragraphs - comment 
                          % this line for an assignment with lots of text

%-------------------------------------------------------------------------------
%   TITLE SECTION
%-------------------------------------------------------------------------------

\newcommand{\horrule}[1]{\rule{\linewidth}{#1}} % horizontal cmd, arg = height
\newcommand{\name}{Colin Bradford} % student name
\newcommand{\hwnum}{2} % homework number
\newcommand{\classnum}{CS 325} % class num with abreviation
\newcommand{\classname}{Analysis of Algorithms} % name of class
\newcommand{\hwtitle}{\classnum: Project \hwnum}

\title{ 
    \normalfont \normalsize 
    \textsc{Oregon State University} \\ [25pt]
    \large Project Group 21
    \horrule{0.5pt} \\[0.4cm] % Thin top horizontal rule
    \huge \hwtitle \\ % The assignment title
    \horrule{2pt} \\[0.5cm] % Thick bottom horizontal rule
}

\author{
    Colin Bradford
    \and
    Charles Jenkins
    \and
    Albert Le
} % Your name

\date{\normalsize\today} % Today's date or a custom date

%-------------------------------------------------------------------------------
%   DOCUMENT
%-------------------------------------------------------------------------------
\begin{document}

\maketitle % Print the title

\begin{enumerate}
    \item The dynamic programming table is filled in by basically marking how
    many coins it takes to get to multiples of a coin value. It is similar to
    taking coin values and saying that certain array positions are "next" to each
    other. For example, if you had a 5-cent coin then you could say that the array
    positions 0, 5, 10, etc. are 1 space away from each other.

    \item Pseudo-code
    \begin{description}
        \item[changeslow]
        \begin{algorithmc}
            \caption{Algorithm 1: changeslow}
            \Require{$A$ is the target change value, and $V$ is an array of all
                     the coin values.}
            \Function{changeslow}{$V$, $A$}
                \State $n \gets V$.size
                \State $coins \gets$ []
                \State $min_m \gets \infty$
                \For{$i \gets 0 \textrm{ to } n$}
                    \If{$V[i] == A$}
                        \State $coins[i] \gets 1$
                        \State $min_m \gets 1$
                        \Return $coins, min_m$
                    \EndIf
                \EndFor
                \For{$i \gets A - 1 \textrm{ to } 0$}
                    \State $C, m \gets $ \Call{changeslow}{$V$, $i$}
                    \State $C2, m2 \gets $ \Call{changeslow}{$V$, $A - i$}
                    \If{$(m + m2) < min_m$}
                        \State $min_m \gets m + m2$
                        \For{$j \gets 0 \textrm{ to } n$}
                            \State $coins[j] \gets C[j] + C2[j]$
                        \EndFor
                    \EndIf
                \EndFor
                \State \Return $coins, min_m$
            \EndFunction
        \end{algorithmc}

        \item[changegreedy]
        \begin{algorithmc}
            \caption{Algorithm 1: changegreedy}
            \Require{$A$ is the target change value, and $V$ is an array of all
                     the coin values.}
            \Function{changegreedy}{$V$, $A$}
                \State $n \gets V$.size
                \State $coins \gets$ []
                \State $min_m \gets \infty$
                \For{$i \gets n \textrm{ to } 0$}
                    \While{$A \geq V[i]$}
                        \State $coins[i] \gets coins[i] + 1$
                        \State $min_m \gets min_m + 1$
                    \EndWhile
                \EndFor
                \Return $coins, min_m$
            \EndFunction
        \end{algorithmc}

        \item[changedp]
        \begin{algorithmc}
            \caption{Algorithm 1: changedp}
            \Require{$A$ is the target change value, and $V$ is an array of all
                     the coin values.}
            \Function{changedp}{$V$, $A$}
                \State $n \gets V$.size
                \State $C \gets$ []
                \State $T \gets$ []
                \State $vals \gets$ []
                \For{$i \gets n \textrm{ to } 0$}
                    \State $k \gets 0$
                    \State $j \gets V[i]$
                    \While{$j \leq A$}
                        \State $k \gets k + 1$
                        \If{$k < T[j]$}
                            \State $T[j] \gets k$
                            \State $coins[j] \gets i$
                        \Else
                            \State $k \gets T[j]$
                        \EndIf
                    \EndWhile
                \EndFor
                \State $j \gets A$
                \While{$j > 0$}
                    \State $i \gets vals[j]$
                    \State $C[i] \gets C[i] + 1$
                    \State $j \gets j - V[i]$
                \EndWhile
                \Return $C, T[A]$
            \EndFunction
        \end{algorithmc}
    \end{description}
\end{enumerate}

\end{document}
%-------------------------------------------------------------------------------


